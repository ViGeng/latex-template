\documentclass{article}

\usepackage{blindtext}
\usepackage{graphicx}
\usepackage{fancyhdr}
\usepackage{hyperref}
\usepackage{changes} % enable revision tracking

% Configure hyperref to handle changes package commands
\hypersetup{
  colorlinks=true,
  linkcolor=blue,
  filecolor=magenta,      
  urlcolor=cyan,
  pdftitle={Report},
  pdfauthor={Author},
  bookmarksdepth=3
}


\usepackage{custom}

\begin{document}

% Abstract
\section*{Abstract}

This is a concise summary of your research, highlighting the key objectives, methods, results, and conclusions. \wei{\gcheck This is a comment from Wei.}

\section{Introduction}

\say[how to use say cmd]{
Basically, if you want to write your paper more structured, every paragraph or several paragraphs is telling a specific concept or idea, you can use the say command to clearly introduce and explain these concepts. For example, I would use a say command in introduction section to introduce the background, a say command to introduce the main research question, and a say command to introduce the methodology.
}

\subsection{Using package changes}

Here is \added{added}, \deleted{deleted} and \replaced{replaced}{replaysed} text. 

\todo[inline]{To-do: Write something worthwhile.}

Here you can introduce the background, motivation, and purpose of your research. \unsure{This is unsure wording.}

I feel sleepy every day. \comment{Maybe I shouldn't have written this?} \unsure{Yes, maybe.}[Maybe I should have.]

\begin{itemize}
  \item Immediate plan of action.
  \begin{todolist}
  	\item[\done] Frame the problem
  	\item Write solution
  	\item[\wontfix] profit
  \end{todolist}
\end{itemize}

\begin{todolist}
  	\item[\done] Frame the problem
  	\item Write solution
  	\item[\wontfix] profit
\end{todolist}

\grayboxtext{a quotation text in a gray box}

\end{document}
